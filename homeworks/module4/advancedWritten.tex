\documentclass[12pt]{article}
\usepackage[top=1in,bottom=1in,left=0.75in,right=0.75in,centering]{geometry}
\usepackage{fancyhdr}
\usepackage{epsfig}
\usepackage[pdfborder={0 0 0}]{hyperref}
\usepackage{palatino}
\usepackage{wrapfig}
\usepackage{lastpage}
\usepackage{color}
\usepackage{ifthen}
\usepackage[table]{xcolor}
\usepackage{graphicx,type1cm,eso-pic,color}
\usepackage{hyperref}
\usepackage{amsmath}
\usepackage{wasysym}
\usepackage{latexsym}
\usepackage{amssymb}

\def\course{CS 4102: Algorithms}
\def\homework{Module}
\def\semester{Spring 2020}

\newboolean{solution}
\setboolean{solution}{false}

% add watermark if it's a solution exam
% see http://jeanmartina.blogspot.com/2008/07/latex-goodie-how-to-watermark-things-in.html
\makeatletter
\AddToShipoutPicture{%
\setlength{\@tempdimb}{.5\paperwidth}%
\setlength{\@tempdimc}{.5\paperheight}%
\setlength{\unitlength}{1pt}%
\put(\strip@pt\@tempdimb,\strip@pt\@tempdimc){%
\ifthenelse{\boolean{solution}}{
\makebox(0,0){\rotatebox{45}{\textcolor[gray]{0.95}%
{\fontsize{5cm}{3cm}\selectfont{\textsf{Solution}}}}}%
}{}
}}
\makeatother

\pagestyle{fancy}

\fancyhf{}
\lhead{\course}
\chead{Page \thepage\ of \pageref{LastPage}}
\rhead{\semester}
%\cfoot{\Large (the bubble footer is automatically inserted into this space)}

\setlength{\headheight}{14.5pt}

\newenvironment{itemlist}{
\begin{itemize}
\setlength{\itemsep}{0pt}
\setlength{\parskip}{0pt}}
{\end{itemize}}

\newenvironment{numlist}{
\begin{enumerate}
\setlength{\itemsep}{0pt}
\setlength{\parskip}{0pt}}
{\end{enumerate}}

\newcounter{pagenum}
\setcounter{pagenum}{1}
\newcommand{\pageheader}[1]{
\clearpage\vspace*{-0.4in}\noindent{\large\bf{Page \arabic{pagenum}: {#1}}}
\addtocounter{pagenum}{1}
\cfoot{}
}

\newcounter{quesnum}
\setcounter{quesnum}{1}
\newcommand{\question}[2][??]{
\begin{list}{\labelitemi}{\leftmargin=2em}
\item [\arabic{quesnum}.] {#2}
\end{list}
\addtocounter{quesnum}{1}
}


\definecolor{red}{rgb}{1.0,0.0,0.0}
\newcommand{\answer}[2][??]{ 
\ifthenelse{\boolean{solution}}{
\color{red} #2 \color{black}}
{\vspace*{#1}}
}

\definecolor{blue}{rgb}{0.0,0.0,1.0}

\begin{document}

\section*{\homework}



%----------------------------------------------------------------------

\question[1]{
Formally prove that the \emph{Bi-Partite Matching} algorithm we saw in class is optimal (i.e, it always find the optimal matching between nodes in the bi-partite graph). \emph{HINT: Assume the algorithm is not optimal and show that you must be able to still find an augmenting path through the network, contradicting your assumption that max-flow terminated.}
}



%----------------------------------------------------------------------


\question[3]{
Delta is trying to convince more customers to become \emph{SUPER DUPER PLATINUM MEMBERS (SDPMs)}, to raise their profits. However, after surveying customers, they discovered that the most attractive feature of the SDPM program is the ability to sit in the \emph{member's lounge} at certain select airports. These lounges offer free food, comfortable seating, and entertainment. What is not to love!!

To make the SDPM program more attractive, Delta wants to put a \emph{member's lounge} in every airport they service but they simply cannot afford it yet. After a long discussion with the executive task force on SPDM member benefits, it is determined that for now, Delta would like to make sure that for all flights, either your starting or ending location (or maybe both) is guaranteed to have a lounge.

Given a graph representing the airports Delta flies to and the flights between them, the \emph{SDPM} problem is to decide whether $k$ \emph{member's lounges} can be placed in order to ensure that each airport or its "neighbor" has a lounge. Show that \emph{SDPM} problem is NP-Complete.

Note: You are not being asked to explicitly solve the \emph{SDPM} problem; you are only required to show that it is NP-Complete.
}

%----------------------------------------------------------------------



\end{document}
